\documentclass{article}
\usepackage{graphicx}
\usepackage{hyperref}
\usepackage[czech]{babel}
\usepackage{csquotes}

\usepackage{biblatex}
\addbibresource{manual.bib}

% Title, Author, Date
\title{PCAP NetFlow v5 exportér}
\author{Martin Slezák (xsleza26)}
\date{\today}

\begin{document}

\maketitle

\newpage

\tableofcontents

\newpage

\section{Základní popis projektu}

Tento projekt implementuje sbírání dat o komunikaci dané PCAP souborem a
exportuje posbírané informace pomocí protokolu Netflow ve verzi 5.

\section{Popis Netflow}

Netflow bylo představeno společností Cisco na jejich routerech okolo roku 1996.
Tato funkcionalita umožňuje sbírat data o IP komunikaci. Když se poté Netflow
analyzuje, síťový administrátor může zjistit různé informace o komunikaci, jako
třeba zdroj a cíl komunikace nebo také kolik paketů v dané komunikaci bylo
posláno.

\subsection{Monitorování}

Monitorování pomocí Netflow je složeno ze tří částí:
\begin{itemize}
    \item \textbf{Flow exporter:} pakety se seskupují do flows a exportují
    \item \textbf{Flow collector:} přijímá flows od exportéru, ukládá a
        předzpracovává
    \item \textbf{Analýza:} analyzuje přijaté flows
        (detekce narušení, profilování,...)
\end{itemize}

\subsection{Hlavička Netflow}

Netflow ve verzi 5 obsahuje na začátku každého paketu hlavičku, která obsahuje
základní informace, jako je třeba verze netflow, ale také například kolik
jednotlivých flow je v daném paketu obsaženo. Konkrétní hodnoty, které jsou
v hlavičce obsaženy jsou:
\begin{itemize}
    \item Verze Netflow
    \item Počet paketů exportovaných v paketu
    \item Aktuální čas v milisekundách od spuštění exportéru
    \item Počet sekund od roku 1970
    \item Počet nanosekund od roku 1970
    \item Celkový počet viděných flows
    \item Typ flow-switching enginu
    \item Číslo slotu flow-switching enginu
    \item První dva bity obsahují mód vzorkování, ostatní interval vzorkování
\end{itemize}

\subsection{Flow identifikátor}
\label{sec:flow-key}

Pro identifikaci flow a seskupování paketů do odpovídající flow se v Netflow
ve verzi 5 využívá klíč, který obsahuje sedm hodnot:
\begin{enumerate}
    \item Rozhraní ingres
    \item Zdrojové IP adresy
    \item Cílové IP adresy
    \item Číslo IP protokolu
    \item Zdrojový port UDP/TCP
    \item Cílový port UDP/TCP
    \item Typ služby IP (ToS)
\end{enumerate}

\subsection{Flow záznam}
\label{sec:flow-record}

Pakety jsou seskupeny do záznamu flow. Flow v Netflow ve verzi 5 obsahuje:

\begin{itemize}
    \item Zdrojovou IP adresu
    \item Cílovou IP adresu
    \item IP adresa nexthop routeru
    \item SNMP index vstupního rozhraní
    \item SNMP index výstupního rozhraní
    \item Počet paketů ve flow
    \item Celkový počet bytů v L3 v paketech flow
    \item Čas prvního paketu od spuštění exportéru
    \item Čas posledního paketu od spuštění exportéru
    \item Zdrojový port TCP/UDP nebo ekvivalent
    \item Cílový port TCP/UDP nebo ekvivalent
    \item Nevyužitý byte
    \item OR TCP flagů všech paketů
    \item Číslo IP protokolu
    \item Typ IP služby (ToS)
    \item Číslo autonomního systému zdroje, buď origin nebo peer
    \item Číslo autonomního systému cíle, buď origin nebo peer
    \item Bity prefix masky zdrojové IP adresy
    \item Bity prefix masky cílové IP adresy
    \item Nevyužité byty
\end{itemize}

\section{Popis implementace}

Tento projekt implementuje pouze jednu z částí monitorování komunikace pomocí
Netflow a to flow exportér.

Implementace flow exporéru by se dala rozdělit do tří hlavních částí:
\begin{itemize}
    \item Přijetí a parsování paketu
    \item Seskupování paketů do flows
    \item Exportování flows
\end{itemize}

\subsection{Přijetí a parsování paketu}

Samotná implementace přijetí a parsování paketu je pomocí knihovny pcap. Ta
umožňuje čtení PCAP souboru a následné čtení jednotlivých paketů. Každý paket
se následně parsuje, aby se získaly všechna potřebná data. Tuto funkcionalitu
přijimání paketu imlementuje metoda parse ve třídě Parser (soubor parser.cpp).

Parsování paketů je v podstatě pouze přetypování bytů dat na odpovídající
strukturu, jako třeba ip, tcphdr. Následně se z těchto struktur dá přečíst
potřebná data.

\subsection{Seskupování paketů do flows}

Z parsovaných dat lze následně sestavit jednotlivé flows. Flow v Netflow ve
verzi 5 je jednoznačně identifikována pomocí klíče, který byl popsán v sekci
\hyperref[sec:flow-key]{flow indentifikátor}.

V samotné implementaci nám ale budou stačit pouze některé hodnoty z výše
uvedené jednoznačné identifikace. Implementovaný exportér podporuje pouze
komunikaci TCP, nepotřebujeme tedy číslo IP protokolu. Nebudeme také potřebovat
rozhraní ingres. Výsledný klíč flow bude tedy vypadat následovně:
\begin{itemize}
    \item Zdrojová IP adresa
    \item Cílová IP adresa
    \item Zdrojový port TCP
    \item Cílový port TCP
    \item Typ služby IP (ToS)
\end{itemize}

Tento klíč je implementován jako struktura FlowKey, která obsahuje jak potřebné
hodnoty, tak i implementuje hašování.

Z přijatého a parsovaného paketu se sestaví výše zmíněný klíč, na základě
kterého se upraví existující flow nebo vytvoří nová v případě, že ještě
neexistuje. Do flow jsou uloženy veškeré informace, které má flow obsahovat
(dle popisu v sekci \hyperref[sec:flow-record]{flow záznam}).

\subsection{Exportování flows}

Flow se exportují v několika případech:
\begin{itemize}
    \item Aktivní časový limit
    \item Neaktivní časový limit
    \item Konec PCAP souboru
\end{itemize}

\subsubsection{Aktivní časový limit}
První případ je po vypršení aktivního časového limitu. Časový limit se
nastavuje při spuštění programu (výchozí hodnota je 60 sekund). Tento časový
limit určuje maximální čas od přijetí prvního paketu v dané flow. Jakmile je
čas od přijetí prvního paketu vyšší než nastavená hodnota, flow je ukončena
a předána k exportu.

\subsubsection{Neaktivní časový limit}
Druhý případ je po vypršení neaktivního časového limitu. Podobně jako u
aktivního časového limitu je při spuštění programu nastavena jeho hodnota
(výchozí je 60 sekund). Neaktivní timeout určuje maximální dobu, kdy nepřijde
žádný paket do dané flow. Jaklmile je čas posledního přijatého paketu vyšší
než nastavená hodnota, flow se ukončí a předá k exportu.

\subsubsection{Konec PCAP souboru}
Poslední případ je dočtení souboru PCAP, kdy chceme odeslat zbylé flows.
Všechny flows, u kterých ještě nevypršel ani jeden z časových limitů se předají
exportéru.

\subsubsection{Exportér}
Exportér jako takový sbírá exportované flows a čeká, dokud nebude mít 29
záznamů flow. Poté vytvoří paket složený z Netflow hlavičky a flow záznamů a
pošle paket na flow collector, kterého adresa a port je nastavena při spuštění
programu. Po přečtení celého PCAP souboru se odešle paket, který nemusí být
plně naplněn.

\section{Návod na použití}
Pro spuštění programu samotného je nejprve potřeba ho kompilovat, to můžeme
udělat spuštěním Makefile (příkaz \texttt{make}).

\section{Testování}
Abych si mohl otestovat, že se Netflow exportuje jak má, tak jsem si
naprogramoval Netflow collector. Tento program očekává pakety na dané adrese
a portu a po přijetí vytiskne informace z Netflow. Díky tomu jsem si mohl
zkontroloval, jestli jednotlivé hodnoty dávají smysl a také, jestli používám
správný endianness.

Výstup tohoto programu jsem ještě zkontroloval s Wiresharkem, kde jsem si
otevřel PCAP soubor, který jsem používal a zkontroloval jsem IP adresy, porty
a ostatní informace, které se daly z Wiresharku přečíst.

Při prvním běhu testu jsem objevil, že jsem zapomněl u některých hodnot řešit
endianness. Po opravě však jednotlivé informace odpovídali informacích z
Wiresharku, tudíž jsem považoval test za úspěšný.

\nocite{*}
\printbibliography[title={Reference}]

\end{document}
